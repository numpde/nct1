\documentclass[12pt,notitlepage]{article}

\usepackage[utf8]{inputenc}
\usepackage[english]{babel}

% https://tex.stackexchange.com/questions/11778/modifying-everydisplay-causes-the-align-environment-to-stop-working
\let\displaystyle\textstyle

\usepackage[
backend=biber,
giveninits=true,
url=false,
isbn=false,
backref=true,
style=alphabetic,
sorting=ynt,
block=none,
maxcitenames=3,
maxbibnames=100,
]{biblatex}
%
\addbibresource{refs.bib}

%\usepackage{natbib} % bibtex
%\let\cite\citep


\usepackage{amsmath, amsfonts, amssymb, mathtools}
\usepackage[svgnames]{xcolor}
\usepackage{datetime2}
\usepackage[
	colorlinks=true, 
	citecolor={DarkRed}, urlcolor={DarkBlue}, linkcolor={DarkBlue},
]{hyperref}


% https://tex.stackexchange.com/questions/3802/how-to-get-doi-links-in-bibliography
% \usepackage{doi}

% 
\usepackage[version=4]{mhchem}
\mhchemoptions{font=sf}

% http://mirrors.ibiblio.org/CTAN/macros/latex/contrib/siunitx/siunitx.pdf
\usepackage{siunitx}


\usepackage{fullpage}

% Paragraph spacing
\usepackage{parskip}
%\usepackage{enumitem}

\usepackage{xspace}

\usepackage{graphicx}
\graphicspath{{..//}{../images/}}
\DeclareGraphicsExtensions{.pdf,.eps,.png}

% https://tex.stackexchange.com/questions/202046/width-of-the-caption-of-a-figure
% https://tex.stackexchange.com/questions/29039/how-to-limit-the-figure-caption-width
% https://tex.stackexchange.com/questions/822/change-the-font-of-figure-captions
\usepackage[margin=10px,font={small}]{caption}
% https://tex.stackexchange.com/questions/25879/multiple-captions-for-a-single-figure
\usepackage{subcaption}


% http://www.texfaq.org/FAQ-ftnsect
\usepackage[stable]{footmisc}

% https://tex.stackexchange.com/questions/20792/how-to-superimpose-latex-on-a-picture
\usepackage[percent]{overpic}

%\usepackage{epstopdf}

% Tables (the order matters here)
\usepackage{makecell}
\usepackage{booktabs}
\usepackage{arydshln}

% https://tex.stackexchange.com/questions/109467/footnote-in-tabular-environment
\usepackage{footnote}
\makesavenoteenv{tabular}
\makesavenoteenv{table}

% https://tex.stackexchange.com/questions/10130/use-the-values-of-title-author-and-date-on-a-custom-title-page
\usepackage{authoraftertitle}

% https://en.wikibooks.org/wiki/LaTeX/Footnotes_and_Margin_Notes#Margin_Notes
\usepackage{marginnote}

% For editing purposes:
%\usepackage[margin=10pt]{geometry}

% https://latex.org/forum/viewtopic.php?t=10456
\usepackage{titlesec}
\titleformat{\subsubsection}[runin]% runin puts it in the same paragraph
{\normalfont\bfseries}% formatting commands to apply to the whole heading
{\thesubsubsection}% the label and number
{}% space between label/number and subsection title
{}% formatting commands applied just to subsection title
[.]% punctuation or other commands following subsection title


\input{todomech}


\renewcommand{\d}{\mathrm{d}}
\newcommand{\ddt}{\frac{\d}{\d{t}}}

\newcommand{\TEXT}[1]{\quad\text{#1}\quad}
\newcommand{\with}{\text{$\,{:}\,$}}

\newcommand{\cbra}[1]{{\ensuremath{\color{gray}{#1}}}}
\newcommand{\signal}[1]{{{\cbra{\langle}\ce{#1}\cbra{\rangle}}}}
\newcommand{\protein}[1]{{{\cbra{(}\ce{#1}\cbra{)}}}}
\newcommand{\promoter}[1]{{{\cbra{[}\ce{#1}\cbra{]}}}}

% https://tex.stackexchange.com/questions/543953/arrow-with-blunted-end-head-in-math-mode
\newcommand{\act}{\ {\ensuremath{\mathbin{\to}}}\ }
\newcommand{\rep}{\ {\ensuremath{\mathrel{\raisebox{-.3ex}{\rotatebox{90}{\scalebox{1}[1.2]{$\bot$}}}}}}\ }

\def\[#1\]{\begin{align}#1\end{align}}

% https://tex.stackexchange.com/questions/114113/how-to-label-text-with-equation-number
\newcommand{\eqnum}{\leavevmode\hfill\refstepcounter{equation}\textup{{(\theequation)}}}

\newcommand{\starlink}[1]{\textsuperscript{\makebox[0pt]{\href{#1}{\color{white}$\star$}}}}

\newcommand{\hh}[1]{{\color{Purple}#1}}
\newcommand{\ra}[1]{{\color{Blue}#1}\marginnote{\TODO{review}}}


\title{DRAFT: NCT}
\author{RA}
\date{\today}
\newcommand{\linktodoc}{http://bit.ly/}




\begin{document}

\maketitle

\section{NCT models}

Abbreviations:
ODE = ordinary differential equations.

\subsection{GSR'03 model of NCT}

First we implement
the ``minimal Ran gradient system'' from 
\cite{GoerlichSeewaldRibbeck2003}.
%
%
The equations are shown in
Table~\ref{t:GSR-Ran}
and
the constants are collected in 
Table~\ref{t:GSR-const}.
%
%
Simulating the ODE
across the scenarios of 
\cite{GoerlichSeewaldRibbeck2003}
we obtain 
results that are fairly close
to the original,
see Table~\ref{t:GSR-Ran-Runs}.
%
%
The ``dynamic capacity'' \ce{Ex}
is an optional maximal (positive) flux
of nuclear \ce{Ran.GTP} to cytoplasmic \ce{Ran.GDP},
which we determine using the additional equation
\[
	\label{e:Ex}
	%
	\ddt \ce{Ex}
	=
	k_{\ce{Ex}} \, [\ce{Ran . {GTP}}]_\text{nuc},
	\quad
	k_{\ce{Ex}} := \SI{10}{s^{-2}},
	\TEXT{initial}
	\ce{Ex} := \SI{0}{\micro M . s^{-1}}.
\]

%

A coupling of the Ran gradient
to 
importin--cargo transport
was proposed in 
\cite[Fig.~6A]{GoerlichSeewaldRibbeck2003}.
%
We formulate (part of) it in 
Table~\ref{t:GSR-Imp}.



\begin{table}
%
The following account for the cytoplasmic species.
%
Here,
$\ce{[\ldots]}$ abbreviates 
the (cytoplasmic) concentration of 
the complex \ce{{RanBP1} . Ran.GTP}.
%(in presence of \ce{RanGAP}, this term can be omitted).
%
%
\begin{subequations}
\[
	\label{e:Eq1}
	%
	\ddt
	[\ce{Ran.GDP}]_\text{cyt}
	& =
	\ce{F_{\ce{Ran.GDP}}} \frac{V_\text{nuc}}{V_\text{cyt}} 
	+
	\ce{GAP}
	+
	\ce{GAP_{RanBP1}}
	+
	\ce{Ex} \frac{V_\text{nuc}}{V_\text{cyt}} 
	%
	%
	\\
	\label{e:Eq2}
	%
	\ddt
	[\ce{Ran.GTP}]_\text{cyt}
	& = 
	\ce{F_{\ce{Ran.GTP}}} \frac{V_\text{nuc}}{V_\text{cyt}}
	-
	\ce{GAP}
	-
	k_\text{on}^\text{rbp}
	[\ce{{RanBP1}}] [\ce{Ran.GTP}]_\text{cyt}
	+
	k_\text{off}^\text{rbp}
	[\ce{\ldots}]
	%
	%
	\\
	\label{e:Eq3}
	%
	\ddt
	[\ce{{RanBP1} . Ran.GTP}]
	& =
	-
	\ce{GAP_{RanBP1}}
	\quad\quad\quad
	+
	k_\text{on}^\text{rbp}
	[\ce{{RanBP1}}] [\ce{Ran.GTP}]_\text{cyt}
	-
	k_\text{off}^\text{rbp}
	[\ce{\ldots}]
\]
\end{subequations}


The following account for the nuclear species.
%
\ce{E} denotes free \ce{{RCC1}}.
%
%
\begin{subequations}
\[
	\label{e:Eq4}
	%
	\ddt
	[\ce{Ran.GDP}]_\text{nuc}
	& =
	-
	\ce{F_{\ce{Ran.GDP}}}
	+
	r_8
	[\ce{IntC}]
	-
	r_1
	[\ce{E}]
	[\ce{Ran.GDP}]_\text{nuc}
	%
	%
	\\
	\label{e:Eq5}
	%
	\ddt
	[\ce{Ran.GTP}]_\text{nuc}
	& =
	-
	\ce{F_{\ce{Ran.GTP}}}
	+
	r_4
	[\ce{IntA}]
	-
	r_5
	[\ce{E}]
	[\ce{Ran.GTP}]_\text{nuc}
	-
	\ce{Ex}
\]
\end{subequations}


The nucleotide-exchange reaction
\ce{Ran.GDP + GTP <=> Ran.GTP + GDP}
is catalyzed by \ce{{RCC1}}.
%
It is modeled as in 
\cite[Fig.~6]{KlebePrinzWittinghoferGoody1995}
/
\cite[Fig.~1]{GoerlichSeewaldRibbeck2003}
with three intermediates.
%
Note that it depends on
the availability of \ce{GDP} and \ce{GTP}.
%
%
\begin{subequations}
\[
	\label{e:Eq6}
	%
	\ddt
	[\ce{IntA}]
	& =
	-(r_4 + r_6)
	[\ce{IntA}]
	+
	r_5
	[\ce{E}] [\ce{Ran.GTP}]_\text{nuc}
	+
	r_3
	[\ce{GTP}] [\ce{IntB}]
	%
	%
	\\
	\label{e:Eq7}
	%
	%
	\ddt
	[\ce{IntB}]
	& =
	r_6 [\ce{IntA}]
	+
	r_2 [\ce{IntC}]
	-
	(r_3 [\ce{GTP}] + r_7 [\ce{GDP}])
	[\ce{IntB}]
	%
	\\
	\label{e:Eq8}
	%
	\ddt
	[\ce{IntC}]
	& =
	-
	(r_2 + r_8) [\ce{IntC}]
	+
	r_1 [\ce{E}] [\ce{Ran.GDP}]_\text{nuc}
	+
	r_7 [\ce{GDP}] [\ce{IntB}]
\]
\end{subequations}


Constraints on the total concentration:
%
%
\begin{subequations}
\[
	\label{e:Eq9}
	%
	\text{Free \ce{{RCC1}}:}
	\quad
	\ce{[E]}
	& =
	\ce{{RCC1}_{total}} - (\ce{[IntA]} + \ce{[IntB]} + \ce{[IntC]})
	%
	\\
	\label{e:Eq10}
	%
	\text{Free}
	\quad
	\ce{[{RanBP1}]}
	& =
	\ce{{RanBP1}_{total}} - \ce{[{RanBP1} . Ran.GTP]}
\]
\end{subequations}


Gradient-driven fluxes from 
the nucleus to the cytoplasm:
%
%
\begin{subequations}
\[
	\label{e:Eq11}
	%
	\ce{F_{Ran.GTP}}
	& =
	D_{\ce{Ran.GTP}}
	\;
	([\ce{Ran.GTP}]_\text{nuc} - [\ce{Ran.GTP}]_\text{cyt})
	%
	\\
	\label{e:Eq12}
	%
	\ce{F_{Ran.GDP}}
	& =
	D_{\ce{Ran.GDP}}
	\;
	([\ce{Ran.GDP}]_\text{nuc} - [\ce{Ran.GDP}]_\text{cyt})
\]
\end{subequations}


\ce{RanGAP} hydrolyzes the $\gamma$-phosphate of \ce{Ran.GTP}.
%
This is more efficient
when \ce{Ran.GTP} is bound to \ce{{RanBP1}}
\cite{BischoffKrebberSmirnovaDongPonstingl1995},
reducing the IC50 seven-fold
\cite[Table~I, p.~1091]{GoerlichSeewaldRibbeck2003}.
%
%
%
\begin{subequations}
\[
	\label{e:Eq13}
	%
	\ce{GAP} 
	& = 
	k_{\ce{GAP}} [\ce{RanGAP}]
	/
	(
		1 + K_{\ce{GAP}} / [\ce{Ran.GTP}]_\text{cyt}
	)
	%
	\\
	\label{e:Eq14}
	%
	\ce{GAP_{RanBP1}} 
	& = 
	k_{\ce{GAP}}' [\ce{RanGAP}]
	/
	(
		1 + K_{\ce{GAP}}' / [\ce{{RanBP1} . Ran.GTP}]
	)
\]
\end{subequations}
%
\caption{%
	The minimal Ran gradient system
	from \cite[Fig.~2]{GoerlichSeewaldRibbeck2003}.
	%
	\ce{Ex} is an additional potentially useful flux of 
	nuclear \ce{Ran.GTP} to cytoplasmic \ce{Ran.GDP},
	set by default to zero.
}
\label{t:GSR-Ran}
\end{table}


\begin{table}
\centering
\small
\begin{tabular}{c|c|c}
%	Eqn & Constants & References
%	\\
	\hline
	%
	%
	\eqref{e:Eq1}
	&
	$V_\text{nuc} = \SI{1.2}{pl}$,
	\quad
	$V_\text{cyt} = \SI{1.8}{pl}$
	& 
	\cite[Table II]{GoerlichSeewaldRibbeck2003}
	\\
	\hline
	%
	%
	\eqref{e:Eq1}
	&
	initial condition
	$[\ce{Ran . GDP}]_\text{cyt} = \SI{5}{\micro M}$
	&
	\cite[Table II]{GoerlichSeewaldRibbeck2003}
	\\
	\hline
	%
	%
	\eqref{e:Eq2}--\eqref{e:Eq3}
	&
	$k_\text{on}^\text{rbp} = \SI{0.3}{\micro M^{-1}.s^{-1}}$,
	\quad
	$k_\text{off}^\text{rbp} = \SI{4e-4}{s^{-1}}$
	&
	\cite[Supp.~Table~A]{GoerlichSeewaldRibbeck2003}
	\\
	\hline
	%
	%
	\eqref{e:Eq4}--\eqref{e:Eq8}
	&
	\makecell{
		$r_1 = \SI{74}{\micro M^{-1} . s^{-1}}$,
		\quad
		$r_8 = \SI{55}{s^{-1}}$
		\\
		$r_7 = \SI{11}{\micro M^{-1} . s^{-1}}$,
		\quad
		$r_2 = \SI{21}{s^{-1}}$
		\\
		$r_3 = \SI{0.6}{\micro M^{-1} . s^{-1}}$,
		\quad
		$r_6 = \SI{19}{s^{-1}}$
		\\
		$r_5 = \SI{100}{\micro M^{-1} . s^{-1}}$,
		\quad
		$r_4 = \SI{55}{s^{-1}}$
	}
	&
	\makecell{
		\cite[Supp.~Table~A]{GoerlichSeewaldRibbeck2003}
		\\
		\cite[Fig.~6]{KlebePrinzWittinghoferGoody1995}
	}
	\\
	\hline
	%
	%
	\eqref{e:Eq6}--\eqref{e:Eq8}
	&
	$[\ce{GTP}] = \SI{500}{\micro M}$,
	\quad
	$[\ce{GDP}] = \SI{1.6}{\micro M}$
	&
	\cite[Table II]{GoerlichSeewaldRibbeck2003}
	\\
	\hline
	%
	%
	\makecell{
		\eqref{e:Eq9} \\ \eqref{e:Eq10}
	}
	&
	\makecell{
		$\ce{{RCC1}_{total}} = \SI{0.7}{\micro M}$
		\\
		$\ce{{RanBP1}_{total}} = \SI{2}{\micro M}$
	}
	&
	\makecell{
		\cite[Supp.~Table~B]{GoerlichSeewaldRibbeck2003}
		\\
		\cite[Fig.~4]{GoerlichSeewaldRibbeck2003}
	}
	\\
	\hline
	%
	%
	\makecell{
		\eqref{e:Eq11} \\ \eqref{e:Eq12}
	}
	&
	\makecell{
		$D_{\ce{Ran . GTP}} = \SI{0.03}{s^{-1}}$
		\\
		$D_{\ce{Ran . GDP}} = \SI{0.12}{s^{-1}}$
	}
	&
	\cite[Table II]{GoerlichSeewaldRibbeck2003}
	\\
	\hline
	%
	%
	\makecell{
		\eqref{e:Eq13} \\ \eqref{e:Eq14}
	}
	&
	\makecell{
		$k_{\ce{GAP}} = \SI{10.6}{s^{-1}}$,
		\quad
		$K_{\ce{GAP}} = \SI{0.7}{\micro M}$
		\\
		$k_{\ce{GAP}}' = \SI{10.8}{s^{-1}}$,
		\quad
		$K_{\ce{GAP}}' = \SI{0.1}{\micro M}$
	}
	&
	\makecell{
		\cite[Supp.~Table~A]{GoerlichSeewaldRibbeck2003}
		\\
		\cite[Table~I]{GoerlichSeewaldRibbeck2003}
	}
	\\
	\hline
	%
	%
	\eqref{e:Eq13}--\eqref{e:Eq14}
	&
	cytoplasmic
	$[\ce{RanGAP}] = \SI{0.7}{\micro M}$
	&
	\cite[Table~II / ST~B]{GoerlichSeewaldRibbeck2003}
	% / \cite[Fig.~4]{GoerlichSeewaldRibbeck2003}
	\\
	\hline
	%
	%
	\TODO{Eq}
	&
	$K_R = \SI{5e-4}{\micro M}$
	&
	\cite[Supp.~Table~A]{GoerlichSeewaldRibbeck2003}
	\\
	\hline
\end{tabular}
%
\caption{%
	Constants
	for the ``standard simulation condition''
	at $\SI{25}{\celsius}$.
	%
	Except for \eqref{e:Eq1},
	all species are initialized to zero at $t = 0$.
}
%
\label{t:GSR-const}
\end{table}


\begin{table}
\centering
\footnotesize
\begin{tabular}{c|c|c|c|c}
	\hline
	%
	Condition
	& 
	\makecell{
		Affected \\ parameters
	}
	&
	\makecell{
		Nuclear
		\\
		RanGTP, \si{\micro M}
	}
	&
	\makecell{
		Cytoplasmic
		\\
		RanGTP, \si{\nano M}
	}
	&
	\makecell{
		Dynamic
		\\
		capacity, \si{\micro M \per s}
	}
	%
	\\
	\hline\hline
	%
	``Standard''
	& 
	See Table~\ref{t:GSR-const} 
	&
	4.26
	(4.3)
	&
	7.75
	(7.7)
	&
	0.59 (0.60)
	%
	\\
	\hline
	%
	Omission of RanBP1
	&
	$\ce{{RanBP1}_{total}} := 0$
	&
	4.27
	(4.3)
	& 
	8.13
	(8.1)
	&
	0.59 (0.60)
	%
	\\
	\hline
	%
	200\% RCC1
	&
	\ce{{RCC1}_{total}}
	&
	3.95
	(4.0)
	& 
	7.17
	(7.1)
	&
	0.59 (0.60)
	%
	\\
	\hline
	%
	50\% RCC1
	&
	\ce{{RCC1}_{total}}
	&
	4.31
	(4.3)
	& 
	7.82
	(7.7)
	&
	0.58 (0.60)
	%
	\\
	\hline
	%
	10\% RCC1
	&
	\ce{{RCC1}_{total}}
	&
	3.59
	(3.6)
	& 
	6.50
	(6.4)
	&
	0.46 (0.48)
	%
	\\
	\hline
	%
	1\% RCC1
	&
	\ce{{RCC1}_{total}}
	&
	1.40
	(1.4)
	& 
	2.52
	(2.5)
	&
	0.075 (0.08)
	%
	\\
	\hline
	%
	GTP:GDP = 500:0
	&
	$[\ce{GDP}] := \SI{0}{\micro M}$
	&
	4.80
	(4.8)
	& 
	8.72
	(8.6)
	&
	0.59 (0.60)
	%
	\\
	\hline
	%
	GTP:GDP = 500:50
	&
	$[\ce{GDP}] := \frac{1}{10} [\ce{GTP}]$
	&
	0.98
	(0.8)
	& 
	1.76
	(1.5)
	&
	0.57 (0.58)
	%
	\\
	\hline
	%
	GTP:GDP = 500:500
	&
	$[\ce{GDP}] := [\ce{GTP}]$
	&
	0.12
	(0.12)
	& 
	0.22
	(0.21)
	&
	0.34 (0.34)
	%
	\\
	\hline
	%
	Saturating NTF2
	% Fig.~3
	&
	$D_{\ce{Ran . GDP}} := \SI{0.48}{s^{-1}}$
	&
	5.12
	(5.1)
	& 
	9.32
	(9.2)
	&
	2.18 (2.2)
	%
	\\
	\hline
	%
	No NTF2
	&
	$D_{\ce{Ran . GDP}} := D_{\ce{Ran . GTP}}$
	&
	2.55
	(2.5)
	& 
	4.60
	(4.5)
	&
	0.15 (0.16)
	%
	\\
	\hline
	%
	200\% RanGAP
	&
	$[\ce{RanGAP}]$
	&
	4.27
	(4.3)
	& 
	3.95
	(3.9)
	&
	0.59 (0.60)
	%
	\\
	\hline
	%
	50\% RanGAP
	&
	$[\ce{RanGAP}]$
	&
	4.26
	(4.3)
	& 
	14.9
	(14)
	&
	0.59 (0.60)
	%
	\\
	\hline
	%
	50\% permeability
	&
	$D_{\ce{Ran . GTP}}$
	&
	4.91
	(4.9)
	& 
	4.44
	(4.4)
	&
	0.59 (--)
	%
	\\
	\hline
	%
	200\% permeability
	&
	$D_{\ce{Ran . GTP}}$
	&
	3.41
	(3.4)
	& 
	12.4
	(12.3)
	&
	0.59 (--)
	%
	\\
	\hline
	%
	400\% permeability
	&
	$D_{\ce{Ran . GTP}}$
	&
	2.46
	(2.5)
	& 
	18.0
	(17.8)
	&
	0.59 (--)
	%
	\\
	\hline
\end{tabular}
\caption{%
	Steady-state concentrations
	for the simulation scenarios
	from \cite[Table~II/III]{GoerlichSeewaldRibbeck2003},
	with
	their results shown in brackets.
	%
	Value for $D_{\ce{Ran . GDP}}$ is
	from \cite[Fig.~3]{GoerlichSeewaldRibbeck2003}.
}
\label{t:GSR-Ran-Runs}
\end{table}


\begin{table}
\begin{subequations}
\[
	%
	\label{e:6}
	%
	\ce{R_{cyt}}
	& :=
	-
	k_\text{on}^\text{R} [\ce{Imp\beta}] [\ce{Ran . GTP}]_\text{cyt}
	+
	k_\text{off}^\text{R} [\ce{Imp\beta . Ran . GTP}]_\text{cyt}
	%
	\\
	\label{e:}
	%
	\ce{C_{cyt}}
	& :=
	-
	k_\text{on}^\text{C}
	[\ce{Imp\beta}]
	[\ce{Cargo}]_\text{cyt}
	%\qquad
	+
	k_\text{off}^\text{C}
	[\ce{Imp\beta . Cargo}]_\text{cyt}
	%
	\\
	\label{e:}
	%
	\ddt
	[\ce{Imp\beta . Ran . GTP}]_\text{cyt}
	& = 
	-
	\ce{R_{cyt}}
	+
	\ce{F_{\ce{Imp\beta . Ran . GTP}}}
	\frac{V_\text{nuc}}{V_\text{cyt}} 
	%
	\\
	\label{e:}
	%
	\ddt
	[\ce{Imp\beta}]_\text{cyt}
	& = 
	%
	+
	\ce{R}_\text{cyt} + \ce{C}_\text{cyt}
	+
	\ce{F_{\ce{Imp\beta}}}
	\frac{V_\text{nuc}}{V_\text{cyt}} 
	%
	\\
	\label{e:4}
	%
	\ddt
	[\ce{Imp\beta . Cargo}]_\text{cyt}
	& = 
	- \ce{C}_\text{cyt}
	+ \ce{F_{\ce{Imp\beta . Cargo}}} \frac{V_\text{nuc}}{V_\text{cyt}}
	%
	\\
	\label{e:5}
	%
	\ddt
	[\ce{Cargo}]_\text{cyt}
	& = 
	+ \ce{C}_\text{cyt}
	+ \ce{F_{\ce{Cargo}}} \frac{V_\text{nuc}}{V_\text{cyt}}
\]

\TODO{Analogous equations for \emph{nuclear}}

\TODO{Diffusion eqn}

\TODO{Add \ce{R} to \ce{Ran . GTP} eqn}

\TODO{Need \ce{Imp\beta . \ce{Ran . {GTP}} + Cargo <=> Imp\beta . Cargo + \ce{Ran . {GTP}}}?}

Following \cite{Catimel} and \cite{Riddick},
we also add
the reaction 
\ce{Imp$\beta$ . Cargo <=> Imp$\beta$^* . Cargo}
\end{subequations}
%
\caption{%
	Eq
}
\label{t:GSR-Imp}
\end{table}

%\url{https://web.archive.org/web/20200221012912/http://atlasgeneticsoncology.org/Genes/GC_RANBP2.html}

%%% BIBLIOGRAPHY %%%
\clearpage
\printbibliography % biblatex
\normalsize




%\clearpage

\SHOWTODOS
%\TODO{are the stars correct?}
%\TODO{acknowledgements, first footnote}
%\TODO{review the references section}
%\TODO{check name spelling (authors and footnote)}


\leavevmode\vfill{\tiny\color{lightgray}\hfill{\DTMnow}}
\end{document}
